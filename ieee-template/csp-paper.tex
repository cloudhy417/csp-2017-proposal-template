%%%%%%%%%%%%%%%%%%%%%%%%%%%%%%%%%%%%%%%%%%%%%%%%%%%%%%%%%%%%%%%%%%%%%%%%%%%%%%%%
%2345678901234567890123456789012345678901234567890123456789012345678901234567890
%        1         2         3         4         5         6         7         8

\documentclass[letterpaper, 10 pt, conference]{ieeeconf}  % Comment this line out if you need a4paper

%\documentclass[a4paper, 10pt, conference]{ieeeconf}      % Use this line for a4 paper

\IEEEoverridecommandlockouts                              % This command is only needed if 
                                                          % you want to use the \thanks command

\overrideIEEEmargins                                      % Needed to meet printer requirements.

% See the \addtolength command later in the file to balance the column lengths
% on the last page of the document

% The following packages can be found on http:\\www.ctan.org
%\usepackage{graphics} % for pdf, bitmapped graphics files
%\usepackage{epsfig} % for postscript graphics files
%\usepackage{mathptmx} % assumes new font selection scheme installed
%\usepackage{times} % assumes new font selection scheme installed
%\usepackage{amsmath} % assumes amsmath package installed
%\usepackage{amssymb}  % assumes amsmath package installed
\usepackage{graphicx}
\usepackage[export]{adjustbox}
\graphicspath{ {images/} }


\title{\LARGE \bf
DuckieGuard
}


\author{Yi-Lun Wu$^{1}$, Chong-Sin Huang$^{2}$, Hao-Yun Chen$^{2}$, Shih-Jie Chang$^{2}$, Chang-Yan Li$^{2}$% <-this % stops a space
\thanks{*This work was supported by the Robotics Master Program in National Chiao Tung University, Taiwan}% <-this % stops a space
\thanks{$^{1}$Yi-Lun Wu is with National Chiao Tung University, Taiwan.
        {\tt\small w86763777.eed04@nctu.edu.tw}}%
\thanks{$^{2}$Chong-Sin Huang, Hao-Yun Chen, Shih-Jie Chang, Chang-Yan Li
        {\tt\small b.d.researcher@ieee.org}}%
}


\begin{document}
\maketitle
\pagestyle{empty}

\section{INTRODUCTION \& MOTIVATION}
There are five sections in this template, please follow the sections and feel free to add sub-sections. 

In this section, you need to give an introduction of your team's project started with a couple of sentences that introduce your topic to your readers. You do not have to give too much detailed information, but you should explain why this project is "important".

You should learn some common usage of LaTex, or you can revise this template to your own project proposal.

\section{SYSTEM ARCHITECTURE \& EQUIPMENTS}
\label{section:system_architecture}
\subsection{SYSTEM ARCHITECTURE}
The system block diagram is showned in Fig.~\ref{figure:system_block_diagram}.
Guard\_node is the core controller in this system. 
It subscribes to object\_recognition\_node for recognition results.
When any unusual object is detected, the guard\_node publishs messages
to lane\_controller\_node and notification\_node which control
duckiebot and deliver notifications respectively; then all notifications are sent to
our http server. User agent can request for notifications from the server.
After user receives notifications,
he can control the duckiebot to do some helpful things via mobile device.

Core techniques of each node:
\begin{itemize}
  \item Guard\_node: Flow control.
  \item Object\_recognition\_node: OpenCV or TensorFlow.
  \item Lane\_controller\_node: Lane following.
  \item Inverse\_kinematics\_node: Use buildin node.
  \item Notification\_node: HTTP request or WebSocket, thread handling.
\end{itemize}

Server side techniques:
\begin{itemize}
  \item Django
  \item HTTP request, WebSocket
\end{itemize}

User agent techniques:
\begin{itemize}
  \item Android
  \item HTTP request, WebSocket
\end{itemize}

\subsection{EQUIPMENTS} 
The system doesn't need extra hardware on duckiebot(Fig.~\ref{figure:duckiebot}),
but need a computer to run the server, and a mobile device to receive notification.

\begin{figure}[h] % h means put this image here
\includegraphics[width=1\columnwidth]{csp2017}
\centering
\caption{System block diagram. Red text are the nodes which should be implemented. Dashed lines ara the optional architecture.}
\label{figure:system_block_diagram}
\end{figure}

\begin{figure}[h] % h means put this image here
\includegraphics[width=0.8\columnwidth]{duckiebot}
\centering
\caption{Duckiebot.}
 \label{figure:duckiebot}
\end{figure}

\section{SPECIFIC AIMS}

(individual part)

Specific aims need to be concrete enough, so that it is clear what will be the expected outcomes of this work. You also need a reasonable scope that you could finish in this semester.

\begin{itemize}
\item Specific Aim 1.
\item Specific Aim 2.
\end{itemize}

\section{APPROACH}

(individual part)

This section should include the methods that you will need in order to reach the specific aims. You could include how you will implement your software/hardware, the design of the algorithms. Some preliminary results will be helpful as well. 

You may also state what experiments you will carry out to convince the readers that you reach the specific aims.

\section{SCHEDULE AND TEAM COLLABORATION}

(individual part)

In this section you should write a estimated timeline of your project. This schedule will be very crucial to keep up progress for your project. If you are in a team, you're encouraged to add other team members' on the timeline. It will better show the coordination of your team.
   

\addtolength{\textheight}{-12cm}   % This command serves to balance the column lengths
                                  % on the last page of the document manually. It shortens
                                  % the textheight of the last page by a suitable amount.
                                  % This command does not take effect until the next page
                                  % so it should come on the page before the last. Make
                                  % sure that you do not shorten the textheight too much.

\bibliographystyle{IEEEtran}
\bibliography{egbib}

\end{document}
